


\documentclass[a4paper,10pt]{article}
\usepackage[utf8x]{inputenc}
% *** CITATION PACKAGE ***
\usepackage{cite}
% *** GRAPHICS RELATED PACKAGES ***
\usepackage[pdftex]{graphicx}
% *** MATH PACKAGES ***
\usepackage[cmex10]{amsmath}
\usepackage{amssymb}
%Appendix
\usepackage{appendix}
\usepackage{color}

%to delete
\newcommand{\parham}[1]{\textsf{\emph{\textbf{\textcolor{green}{#1}}}}} 
\newcommand{\dean}[1]{\textsf{\emph{\textbf{\textcolor{yellow}{#1}}}}} 

%opening
\title{Nonlinear Spatio-Temporal Modelling using the Integro-Difference Equation}
\author{Parham Aram*,
        Dean~R.~Freestone,Kenneth Scerri, Michael Dewar,\\ Andrew Zammit Mangion and Visakan Kadirkamanathan}

\begin{document}

\maketitle

\begin{abstract}
The aim of this paper is to 	
\begin{enumerate}
	\item Derive the temporally second order model
	\item Apply the EM algorithm on test data generated by the model above
	\item Extend the EM algorithm to estimate also the model covariances
	\item Apply method above to measured data
\end{enumerate}


\end{abstract}

\section{Introduction}
\begin{enumerate}
	\item General background and motivation
	\item Aims and hypotheses
	\item Specific background
	\item Point out specific things we are doing and why they are important
	\item Paper layout
\end{enumerate}
\section{State-Space Representation of the Spatio-temporal Nonlinear IDE}
The $p^{th}$-order nonlinear, stochastic, spatio-temporal homogenous IDE model is defined as
 \begin{equation}\label{eq:IDEModel}
  \rho_t(\mathbf r)=\int_{\Omega}w(\mathbf r-\mathbf r')f(v_{t-p}(\mathbf r'))d\mathbf r'+\mathbf e_{t-p}(\mathbf r). 
 \end{equation}
The quantity $\rho_t(\mathbf r)$ is a lag polynomial defined as
\begin{equation}\label{eq:LagPolynomial}
 \rho_t(\mathbf r)=(1+\sum_{i=1}^{p}\varphi_i L_i)v_{t}\left(\mathbf r\right),
\end{equation}
where $v_t(\mathbf r)$ denotes a spatio-temporal field at time $t$ and at spatial location $\mathbf r$, $f(\cdot)$ is a nonlinear map, and $w(\cdot)$ is a spatial mixing kernel. The disturbance, $\mathbf e_{t-p}(\mathbf r)$, is a zero-mean normally distributed noise process that is spatially
coloured and temporally white, with covariance
\begin{equation}
cov\left(e_{t}\left(\mathbf{r}\right),e_{t+\tau}\left(\mathbf{r'}\right)\right)=
\gamma\left(\mathbf{r}-\mathbf{r'}\right)\delta(t-\tau)
\label{eq:FieldCovariance}
\end{equation}
and $\delta(\cdot)$ is the Dirac-delta function.
Observations from the spatio-temporal field are described by the measurement function
\begin{equation}
    \label{eq:ObservationEquation}
	\mathbf{y}_t =
	\int_{\Omega}{
	    m\left(\mathbf{r}_{n_y}-\mathbf{r}'\right)v_t\left(\mathbf{r}'\right)
	d\mathbf{r}'} + 
	\boldsymbol{\varepsilon}_t, 
\end{equation}
where the observation vector, $\mathbf{y}_{t}$, compiled at $n_{y}$ spatial locations and at time $t$, is corrupted by an i.i.d normally distributed zero-mean white noise ${\boldsymbol\varepsilon}_t\sim \mathcal{N}\left(\mathbf{0},\boldsymbol\Sigma_{\varepsilon}\right)$ with $\mathbf{\Sigma}_{\boldsymbol\varepsilon}=\sigma_{\varepsilon}^2\mathbf I_{n_y} $. It is assumed here that $\boldsymbol\Sigma_{\varepsilon}$ and $\gamma\left(\mathbf r-\mathbf r' \right)$ are known. The output kernel $m(\mathbf{r}-\mathbf{r}')$ governs the sensor pick-up geometry and is defined by 
\begin{equation}\label{eq:SensorKernel}
	m\left(\mathbf{r}-\mathbf{r}'\right) = \exp{\left(-\frac{(\mathbf{r}-\mathbf{r}')^\top(\mathbf{r}-\mathbf{r}')}{\sigma_m^2}\right)},
\end{equation}
where $\sigma_m$ sets the sensor width. The superscript $\top$ denotes the transpose operator. We proceed by writing the spatio-temporal IDE model in a parameterized state space form which allows the application of standard estimation techniques. We use a decomposition of the field and the spatial mixing kernel using a set of Gaussian basis functions
 \begin{equation}
	\label{eq:FieldDecomp} v_t\left(\mathbf{r}\right) \approx \boldsymbol{\phi}^{\top}\left(\mathbf{r}\right) \mathbf{x}_t
\end{equation}
\begin{equation}\label{eq:KernelDecomp}
	 w\left(\mathbf{r}-\mathbf{r}'\right) \approx\boldsymbol{\psi}^\top\left(\mathbf{r}-\mathbf{r}'\right) \boldsymbol{\theta}
\end{equation}
where $\mathbf{\boldsymbol{\phi}}(\mathbf{r})$ and $\mathbf{\boldsymbol{\psi}}(\mathbf{r})$ are  vectors of Gaussian basis functions that are scaled by the state vector, $\mathbf{x}_t$ and the scaling parameter vector $\boldsymbol\theta$ respectively. Substituting \eqref{eq:LagPolynomial} in \eqref{eq:IDEModel} and using the above decomposition we have
\begin{equation}\label{eq:DecompModel}
\boldsymbol{\phi}^{\top}\left(\mathbf{r}\right)(1+\sum_{i=1}^{p}\varphi_i L_i) \mathbf{x}_t=\int_{\Omega}\boldsymbol{\psi}^\top\left(\mathbf{r}-\mathbf{r}'\right) f(\boldsymbol{\phi}^{\top}\left(\mathbf{r'}\right) \mathbf{x}_{t-p})d\mathbf r'\boldsymbol{\theta}+\mathbf e_{t-p}(\mathbf r)
 \end{equation}
multiplying \eqref{eq:DecompModel} by $\boldsymbol \phi(\mathbf r)$ and integrate over the spatial domain $\Omega$ we get
\begin{equation}\label{eq:GammaDecompModel}
 \boldsymbol \Gamma (1+\sum_{i=1}^{p}\varphi_i L_i) \mathbf{x}_t=\int_{\Omega}\boldsymbol \phi(\mathbf r)\int_{\Omega}\boldsymbol{\psi}^\top\left(\mathbf{r}-\mathbf{r}'\right) f(\boldsymbol{\phi}^{\top}\left(\mathbf{r'}\right) \mathbf{x}_{t-p})d\mathbf r'\boldsymbol{\theta}+\int_{\Omega}\boldsymbol\phi(\mathbf r)\mathbf e_{t-p}(\mathbf r)d\mathbf r
\end{equation}
where 
\begin{equation}\label{eq:DefGamma}
	\boldsymbol{\Gamma} \triangleq \int_\Omega {\boldsymbol{\phi} \left(\mathbf{r}\right)\boldsymbol{\phi} ^{\top}\left(\mathbf{r}\right)d\mathbf{r}} 
\end{equation}
cross-multiplying \eqref{eq:GammaDecompModel} by $\boldsymbol{\Gamma}^{-1}$ gives
\begin{equation}\label{eq:InvGammaDecompModel}
 (1+\sum_{i=1}^{p}\varphi_i L_i) \mathbf{x}_t=\boldsymbol\Gamma^{-1}\int_{\Omega}\boldsymbol \phi(\mathbf r)\int_{\Omega}\boldsymbol{\psi}^\top\left(\mathbf{r}-\mathbf{r}'\right) f(\boldsymbol{\phi}^{\top}\left(\mathbf{r'}\right) \mathbf{x}_{t-p})d\mathbf r'\boldsymbol{\theta}+\boldsymbol\Gamma^{-1}\int_{\Omega}\boldsymbol\phi(\mathbf r)\mathbf e_{t-p}(\mathbf r)d\mathbf r
\end{equation}
This can be further simplified by assuming isotropic structure for the spatial kernel
\begin{equation}
	\boldsymbol{\psi} (\mathbf{r}-\mathbf{r}') = \boldsymbol{\psi} (2c+\mathbf{r}'-\mathbf{r}).
\end{equation}
where $c$ is the center of the kernel. \dean{Can't use c as it is used in the observation matrix.} To make the simplification, we first define
\begin{equation}\label{eq:DefPsi}
	\boldsymbol{\Psi}(\mathbf{r}') \triangleq T_s\boldsymbol{\Gamma}^{-1}\int_\Omega {\boldsymbol{\phi}(\mathbf{r})\boldsymbol{\psi}^{\top} (2c+\mathbf{r}'-\mathbf{r})\textrm{d}\mathbf{r}},
\end{equation}
which is a constant $n_x \times n_{\theta}$ matrix and can be computed analytically. Now substituting \eqref{eq:DefPsi} into \eqref{eq:InvGammaDecompModel} we get
\begin{equation}
(1+\sum_{i=1}^{p}\varphi_i L_i) \mathbf{x}_t = \int_\Omega \boldsymbol{\Psi}(\mathbf{r}') f(\boldsymbol{\phi}^{\top}(\mathbf{r}')\mathbf{x}_t) \textrm{d}\mathbf{r}' \boldsymbol{\theta}
+ \boldsymbol{\Gamma}^{-1} \int_\Omega{\boldsymbol{\phi}(\mathbf{r})e_{t-p}(\mathbf{r})\textrm{d}\mathbf{r}}.
\end{equation}
Now we define the state disturbance as
\begin{equation}\label{eq:Wt} 
	\mathbf{e}_{t-p} \triangleq \boldsymbol{\Gamma}^{-1}\int_\Omega {\boldsymbol{\phi} ( \mathbf{r} )e_{t-p}( \mathbf{r} )\textrm{d}\mathbf{r}},
\end{equation}
which is a zero mean normally distributed white noise process with covariance
\begin{equation}
	\boldsymbol\Sigma_e =\mathbf{\Gamma}^{-1}\int_{\Omega}\int_{\Omega}\boldsymbol{\phi}\left(\mathbf r\right) \gamma\left(\mathbf r- \mathbf r' \right)\boldsymbol{\phi}\left(\mathbf r'\right)^{\top}d\mathbf r' d\mathbf r\mathbf{\Gamma}^{- \top}. 
\end{equation}
the observation equation of the reduced model is found by substituting \eqref{eq:FieldDecomp} into \eqref{eq:ObservationEquation} giving
\begin{equation}\label{ObservationEquation} 
	\mathbf{y}_t = \mathbf{C}\mathbf{x}_t + \boldsymbol{\varepsilon}_t,
\end{equation}
where the observation matrix is 
\begin{equation}
	\mathbf{C} = \left[
	\begin{array}{{ccc}} 
		c_{1,1} & \dots & c_{1,L} \\
		\vdots & \ddots & \vdots \\
		c_{N,1} & \dots & c_{N,L} 
	\end{array}
	\right] 
\end{equation}
and 
\begin{equation}
	c_{i,j} = \int_{\Omega}m(\mathbf{r}_i - \mathbf{r}')\boldsymbol{\phi}_j(\mathbf{r}')\textrm{d}\mathbf{r}'. 
\end{equation}
Now we have the final form of the state-space model where
\begin{equation}\label{eq:finalformstatespacemodel}
	(1+\sum_{i=1}^{p}\varphi_i L_i) \mathbf{x}_t = \int_\Omega \boldsymbol{\Psi}(\mathbf{r}') f(\boldsymbol{\phi}^{\top}(\mathbf{r}')\mathbf{x}_{t-p}) \textrm{d}\mathbf{r}' \boldsymbol{\theta}
 +\mathbf{e}_{t-p}
\end{equation}
\begin{equation} 
	\mathbf{y}_t = \mathbf{C}\mathbf{x}_t + \boldsymbol{\varepsilon}_t
\end{equation}


Here we consider the case where $p=1$, therefore we have
\begin{equation}
x_t+\varphi_{1} x_{t-1}=\boldsymbol\Gamma^{-1}\int_{\Omega}\boldsymbol \phi(\mathbf r)\int_{\Omega}\boldsymbol{\psi}^\top\left(\mathbf{r}-\mathbf{r}'\right) f(\boldsymbol{\phi}^{\top}\left(\mathbf{r'}\right) \mathbf{x}_{t-1})d\mathbf r'\boldsymbol{\theta}+\boldsymbol\Gamma^{-1}\int_{\Omega}\boldsymbol\phi(\mathbf r)\mathbf e_{t-1}(\mathbf r)d\mathbf r
\end{equation}
\section{Deterministic Second-Order Neural Field Model}
\begin{equation}\label{SpikesToPotential}
	v\left( {\mathbf{r},t} \right) = \int_{ - \infty }^t {h\left( {t - t'} \right)g\left( {\mathbf{r},t'} \right)dt'}
\end{equation}
The post-synaptic response kernel $h(t)$ is described by
\begin{equation}\label{SynapticRespKernel}
	h(t) = \eta(t)At\exp{\left(-\zeta t\right)}.
\end{equation}
where $\zeta=\tau^{-1}$ and $\tau$ is the synaptic time constant and $\eta(t)$ is the Heaviside step function. $A$ determines the maximal amplitude of the post-synaptic potentials.
Nonlocal interactions between cortical populations are described by	
\begin{equation}\label{RateBasedInteractions}
	g\left( \mathbf{r},t \right) = \int_\Omega  {w\left( \mathbf{r},\mathbf{r}' \right)f\left( v\left( \mathbf{r}',t \right) \right)d\mathbf{r}'} 
\end{equation}
where $f(.)$ is the firing rate function, $w(.)$ is the spatial connectivity kernel, and $\Omega$ is the spatial domain.
\subsection*{Deriving Integro-differential Equation}
we can write
\begin{eqnarray}\label{SynapticRespKernelderivation}
 \frac{dh(t)}{dt}&=&A[\delta(t)t\mathrm{exp}(-\zeta t)+\eta(t)\frac{d}{dt}(t\mathrm{exp}(-\zeta t))] \nonumber\\
&=& A[0+\eta(t)[\mathrm{exp}(-\zeta t)-\zeta t\mathrm{exp}(-\zeta t) ]\nonumber \\
&=&A\eta(t)\mathrm{exp}(-\zeta t)-\zeta h(t)
\end{eqnarray}
and taking derivative from \ref{SynapticRespKernelderivation} we have
\begin{eqnarray}\label{SynapticRespKernelSecondderivation}
 \frac{d^2h(t)}{dt^2}&=&A[\delta(t)\mathrm{exp}(-\zeta t)-\zeta\eta(t)\mathrm{exp}(-\zeta t)]-\zeta \frac{dh(t)}{dt} \nonumber\\
&=& A\delta(t)-\zeta A \eta(t)\mathrm{exp}(-\zeta t)-\zeta \frac{dh(t)}{dt} \nonumber \\
&=&A\delta(t)-2\zeta A \eta(t)\mathrm{exp}(-\zeta t)+\zeta^2h(t)
\end{eqnarray}
We define the operator $D$ as $D=\frac{d^2}{dt^2}+2\zeta \frac{d}{dt}+\zeta^2 $, applying $D$ on $h$ and using the result from (\ref{SynapticRespKernelSecondderivation}) it is clear that 
\begin{equation}
 Dh=A\delta(t)
\end{equation}
by applying $D$ on (\ref{SpikesToPotential}) we have
\begin{eqnarray}
 Dv&=&D(h*g) \nonumber \\
&=&(Dh*g) \nonumber\\
&=& A(\delta(t)*g)\nonumber \\
&=&Ag
\end{eqnarray}
therefore we have
\begin{equation}\label{eq:finalcontinouosmodel}
\begin{split}
 \frac{d^2v(\mathbf r,t)}{dt^2}+2\zeta \frac{dv(\mathbf r,t)}{dt}+\zeta^2v(\mathbf r,t)&=Ag\left( \mathbf{r},t \right)\\
&=A\int_\Omega  {w\left( \mathbf{r},\mathbf{r}' \right)f\left( v\left( \mathbf{r}',t \right) \right)d\mathbf{r}'} \\
\end{split}
\end{equation}
\subsection*{Discretization}
The above second-order differential equation can be split into two, coupled first order ODEs.\\
\begin{equation}\label{eq:systemofode}
\begin{cases}
\tilde{v}=\frac{dv}{dt} \\
\frac{d\tilde{v}}{dt}+2\zeta\tilde{v}+\zeta^2v=Ag\left( \mathbf{r},t \right)
\end{cases}
\end{equation}
Now we approximate the two first-order ODE by Euler's method
\begin{equation}\label{eq:discritizedsystemofode}
\begin{cases}
v_{t+1}=T_s\tilde{v_t}+v_t\\
\tilde{v}_{t+1}=(1-2\zeta T_s)\tilde{v}_t-\zeta^2T_sv_t+AT_sg_t\left( \mathbf{r}\right)
\end{cases}
\end{equation}
in matrix format we have
\begin{equation}
 \begin{bmatrix}v_{t+1} \\ \tilde{v}_{t+1}\end{bmatrix}= \begin{bmatrix} T_s &1\\ 1-2\zeta T_s & -\zeta^2T_s\end{bmatrix}\begin{bmatrix}v_{t} \\ \tilde{v}_{t}\end{bmatrix}+AT_sg_t\left( \mathbf{r}\right)
\end{equation}
or alternatively we can write (\ref{eq:discritizedsystemofode}) as
\begin{equation}\label{eq:DifferenceModel}
 v_{t+2}-2(1-\zeta T_s)v_{t+1}+(1-2\zeta T_s+\zeta^2T_s^2)v_t=AT_s^2g_t\left( \mathbf{r}\right)
\end{equation}
\subsection*{From Difference Equation to State-Space Model}
We define
\begin{equation}\label{eq:tildev}
v_{t+1}-v_t=T_s\tilde{v_t}
\end{equation}
expanding \eqref{eq:DifferenceModel} we get
\begin{equation}\label{eq:rearrangedDifferenceModel}
 v_{t+2}-2v_{t+1}+2\zeta T_S v_{t+1}+v_t-2\zeta T_sv_t+\zeta^2T_s^2v_t=AT_s^2g_t\left( \mathbf{r}\right)
\end{equation}
rearranging \eqref{eq:rearrangedDifferenceModel} we have
\begin{equation}
(v_{t+2}-v_{t+1})-(v_{t+1}-v_t)+2\zeta T_s(v_{t+1}-v_t)+\zeta^2T_s^2v_t=AT_s^2g_t\left( \mathbf{r}\right)
\end{equation}
therefore
\begin{equation}
(v_{t+2}-v_{t+1})-(1-2\zeta T_s)(v_{t+1}-v_t)+\zeta^2T_s^2v_t=AT_s^2g_t\left( \mathbf{r}\right)
\end{equation}
substituting \eqref{eq:tildev} in the above equation and multiplying by $1/T_s$ we get the state space model
\begin{equation}\label{eq:StateSpaceModel}
\begin{cases}
v_{t+1}=T_s\tilde{v_t}+v_t\\
\tilde{v}_{t+1}=(1-2\zeta T_s)\tilde{v}_t-\zeta^2T_sv_t+AT_sg_t\left( \mathbf{r}\right)
\end{cases}
\end{equation}

\section{First order stochastic Neural Field Model}
The model relates the average number of action potentials $g(\mathbf{r},t)$ arriving at time $t \in \mathbb R^{+}$ and at position $\mathbf{r}$ to the local post-synaptic membrane voltage $v(\mathbf{r},t)$. The post-synaptic potentials generated at a neuronal population at location $\mathbf{r}$ by action potentials arriving from all other connected populations at locations $\mathbf{r}'$ is described by 
\begin{equation}
	\label{eq:SpikesToPotential} v\left( {\mathbf{r},t} \right)=v(\mathbf r, 0)h(t)+\int_0^t {h\left( {t - t'} \right)g\left( {\mathbf{r},t'} \right)dt'} 
\end{equation}
The post-synaptic response kernel $h(t)$ is described by 
\begin{equation}
	\label{eq:SynapticRespKernel} h(t) = \exp{\left(-\zeta t\right)} 
\end{equation}
where $\zeta=\tau^{-1}$, $\tau$ is the synaptic time constant.
Let $W(\mathbf r,t)$ be a spatio-temporal Wiener process. Then, for $\sigma \in \mathbb R^{+}$, the non-local interactions $g\left( {\mathbf{r},t} \right)$ between cortical populations can be divided into a predominantly deterministic part $g'\left( {\mathbf{r},t} \right) $ and  the random component $W(\mathbf r,t)$,
\begin{equation}\label{eq:ActPotComponents}
  g\left( {\mathbf{r},t} \right)dt=g'\left( {\mathbf{r},t} \right)dt+\sigma d W(r,t)
\end{equation}
As a result, the post-synaptic membrane voltage $v(\mathbf r, t)$ is itself a stochastic process. The deterministic part $g'(\mathbf r, t)$ is usually modelled as an integral equation with exogenous deterministic input , $ u(\mathbf r,t)$, of the form \cite{Atay2005}
\begin{equation}
	\label{DeterministicRateBasedInteractions} g'\left( \mathbf{r},t \right) = \int_\Omega {w\left( \mathbf{r},\mathbf{r}' \right)f\left( v\left( \mathbf{r}',t \right) \right)\textrm{d}\mathbf{r}'}+u(\mathbf r,t)
\end{equation}
where $f(\cdot)$ is the firing rate function, $w(\cdot)$ is the spatial connectivity kernel and $\Omega$ is the spatial domain representing a cortical sheet or surface. Substituting \eqref{eq:ActPotComponents} in \eqref{eq:SpikesToPotential} we get
\begin{equation}\label{eq:NeuralModelDeterStoch}
v\left( {\mathbf{r},t} \right)=v_0h(t)+\int_0^t {h\left( {t - t'} \right)g'\left( {\mathbf{r},t'} \right)dt'}+ \int_0^t h\left( {t - t'} \right)\sigma d W(r,t')
\end{equation}
Here we assume that $v_0$ is generated by a known distribution $p_0(v)$ and is independent of the the random component $W(\mathbf r,t)$. We stress that $\sigma$ does not depend on the field $v(\mathbf r, t)$
and hence the noise in \eqref{eq:NeuralModelDeterStoch} is strictly additive. It can be shown that the post-synaptic potential $v(\mathbf r, t)$ solves the following initial value problem (see Appendix \ref{app:InitialValueProblemProof})
\begin{equation}
 dv(\mathbf r, t) + \zeta v(\mathbf r, t)dt = g'(\mathbf r, t)dt + dW(\mathbf r, t) \quad t\ge0, v(\mathbf r, 0) = v_0
\end{equation}
applying the Euler-Maruyama method the integro-difference equation (IDE) model takes the form of
\begin{equation}\label{eq:Euler-MaruyamaDiscrit}
 v_{t+T_s}(\mathbf r)- v_{t}(\mathbf r)+ T_s\zeta v_t(\mathbf r)= T_sg'(\mathbf r, t) + (W_{t+T_s}(\mathbf r)-W_t(\mathbf r))
\end{equation}
where $T_s$ is the time step, the increments $W_{t+T_s}(\mathbf r)-W_t(\mathbf r)$ is normally distributed with mean zero and covariance function $T_s\gamma(\mathbf r-\mathbf r')$. We rewrite \eqref{eq:Euler-MaruyamaDiscrit} as
\begin{equation}
	\label{DiscreteTimeModel} 
	v_{t+T_s}\left(\mathbf{r}\right) = 
	\xi v_t\left(\mathbf{r}\right) + 
	T_s \int_\Omega { 
	    w\left(\mathbf{r},\mathbf{r}'\right)
	    f\left(v_t\left(\mathbf{r}'\right)\right) 
	\textrm{d}\mathbf{r}'}+T_su_t(\mathbf r)
	+ e_t\left(\mathbf{r}\right), 
\end{equation}
 where $\xi = 1-\zeta T_s$ and $e_t(\mathbf{r})$ is an $i.i.d.$ disturbance such that $e_t(\mathbf{r})\sim\mathcal{GP}(\mathbf 0,T_s\gamma(\mathbf{r}-\mathbf{r}'))$. Here $\mathcal{GP}(\mathbf 0,T_s\gamma(\mathbf{r}-\mathbf{r}'))$ denotes a zero mean spatial Gaussian process with covariance function $T_s\gamma(\mathbf{r}-\mathbf{r}')$. To simplify the notation, the index of the future time sample, $t+T_s$, shall be referred to as $t+1$ throughout the rest of the paper. 
\section{Second order stochastic Neural Field Model}
The model relates the average number of action potentials $g(\mathbf{r},t)$ arriving at time $t \in \mathbb R^{+}$ and at position $\mathbf{r}$ to the local post-synaptic membrane voltage $v(\mathbf{r},t)$. The post-synaptic potentials generated at a neuronal population at location $\mathbf{r}$ by action potentials arriving from all other connected populations at locations $\mathbf{r}'$ is described by 
\begin{equation}
	\label{eq:SecondOrderSpikesToPotential} v\left( {\mathbf{r},t} \right)=v(\mathbf r, 0)(t+1)e^{-\zeta t}+v'(\mathbf r,0)te^{-\zeta t}+\int_0^t {h\left( {t - t'} \right)g\left( {\mathbf{r},t'} \right)dt'} 
\end{equation}
The post-synaptic response kernel $h(t)$ is described by 
\begin{equation}
	\label{eq:SecondOrderSynapticRespKernel} h(t) = At\exp{\left(-\zeta t\right)} 
\end{equation}
where $\zeta=\tau^{-1}$, $\tau$ is the synaptic time constant.  $A$ determines the maximal amplitude of the post-synaptic potentials.
Let $W(\mathbf r,t)$ be a spatio-temporal Wiener process. Then, for $\sigma \in \mathbb R^{+}$, the non-local interactions $g\left( {\mathbf{r},t} \right)$ between cortical populations can be divided into a predominantly deterministic part $g'\left( {\mathbf{r},t} \right) $ and  the random component $W(\mathbf r,t)$,
\begin{equation}\label{eq:SecondOrderActPotComponents}
  g\left( {\mathbf{r},t} \right)dt=g'\left( {\mathbf{r},t} \right)dt+\sigma d W(r,t)
\end{equation}
As a result, the post-synaptic membrane voltage $v(\mathbf r, t)$ is itself a stochastic process. The deterministic part $g'(\mathbf r, t)$ is usually modelled as an integral equation with exogenous deterministic input , $ u(\mathbf r,t)$, of the form \cite{Atay2005}
\begin{equation}
	\label{DeterministicRateBasedInteractions} g'\left( \mathbf{r},t \right) = \int_\Omega {w\left( \mathbf{r},\mathbf{r}' \right)f\left( v\left( \mathbf{r}',t \right) \right)\textrm{d}\mathbf{r}'}+u(\mathbf r,t)
\end{equation}
where $f(\cdot)$ is the firing rate function, $w(\cdot)$ is the spatial connectivity kernel and $\Omega$ is the spatial domain representing a cortical sheet or surface. Substituting \eqref{eq:SecondOrderActPotComponents} in \eqref{eq:SecondOrderSpikesToPotential} we get
\begin{equation}\label{eq:SecondOrderNeuralModelDeterStoch}
v\left( {\mathbf{r},t} \right)=v(\mathbf r, 0)(t+1)e^{-\zeta t}+v'(\mathbf r,0)te^{-\zeta t}+\int_0^t {h\left( {t - t'} \right)g'\left( {\mathbf{r},t'} \right)dt'}+ \int_0^t h\left( {t - t'} \right)\sigma d W(r,t')
\end{equation}
It can be shown that the post-synaptic potential $v(\mathbf r, t)$ solves the following initial value problem \parham{Andrew do you know how?}
\begin{equation*}\label{eq:SecondOrderInitialValueProblem}
\left\lbrace \begin{array}{lc}
dv(\mathbf r,t)=\tilde{v}(\mathbf r,t)dt & \\
d\tilde{v}(\mathbf r,t)+2\zeta\tilde{v}(\mathbf r,t)dt+\zeta^2v(\mathbf r,t)dt=&g'(\mathbf r,t)dt+dW(\mathbf r,t)
\end{array}\right.
\end{equation*}
the linear second order SDE can be written in matrix notation as
\begin{equation}\label{eq:SecondOrderSDEMatrix}
 d\begin{bmatrix} v_t(\mathbf r ,t) \\ \tilde{v}_t(\mathbf r ,t)\end{bmatrix}=\begin{bmatrix}0 & 1 \\ -\zeta^2 & -2\zeta \end{bmatrix}\begin{bmatrix} v_t(\mathbf r ,t) \\ \tilde{v}_t(\mathbf r ,t)\end{bmatrix}dt+\begin{bmatrix}0 \\ g'(\mathbf r, t)\end{bmatrix}dt+\begin{bmatrix}0 \\ 1 \end{bmatrix}dW(\mathbf r,t)
\end{equation}
applying the Euler-Maruyama method the integro-difference equation (IDE) model takes the form of
\begin{equation}\label{eq:SecondOrderEulerMaruyamaDis}
 \begin{bmatrix} v_{t+T_s}(\mathbf r)-v_t(\mathbf r) \\ \tilde{v}_{t+T_s}(\mathbf r)-\tilde{v}_t(\mathbf r)\end{bmatrix}=T_s\begin{bmatrix}0 & 1 \\ -\zeta^2 & -2\zeta \end{bmatrix}\begin{bmatrix} v_t(\mathbf r)\\ \tilde{v}_t(\mathbf r)\end{bmatrix}+T_s\begin{bmatrix}0 \\ g'(\mathbf r, t)\end{bmatrix}+\begin{bmatrix}0 \\ 1 \end{bmatrix}(W_{t+T_s}(\mathbf r)-W_t(\mathbf r))
\end{equation}
where $T_s$ is the time step, the increments $W_{t+T_s}(\mathbf r)-W_t(\mathbf r)$ is normally distributed with mean zero and covariance function $T_s\gamma(\mathbf r-\mathbf r')$. We rewrite \eqref{eq:SecondOrderEulerMaruyamaDis} as
\begin{equation}
 \begin{bmatrix} v_{t+T_s}(\mathbf r) \\ \tilde{v}_{t+T_s}(\mathbf r)\end{bmatrix}=\begin{bmatrix}1 & T_s \\ -\zeta^2 T_s & 1-2\zeta T_s \end{bmatrix}\begin{bmatrix} v_t(\mathbf r)\\ \tilde{v}_t(\mathbf r)\end{bmatrix}+T_s\begin{bmatrix}0 \\ g'(\mathbf r, t)\end{bmatrix}+\begin{bmatrix}0 \\ 1 \end{bmatrix}e_t(\mathbf r)
\end{equation}
 where $e_t(\mathbf{r})$ is an $i.i.d.$ disturbance such that $e_t(\mathbf{r})\sim\mathcal{GP}(\mathbf 0,T_s\gamma(\mathbf{r}-\mathbf{r}'))$. Here $\mathcal{GP}(\mathbf 0,T_s\gamma(\mathbf{r}-\mathbf{r}'))$ denotes a zero mean spatial Gaussian process with covariance function $T_s\gamma(\mathbf{r}-\mathbf{r}')$. To simplify the notation, the index of the future time sample, $t+T_s$, shall be referred to as $t+1$ throughout the rest of the paper. 
\section{Estimation of the Non-linear Spatio-temporal IDE Model}

\subsection{Unscented Kalman Filter}
\subsection{Unscented RTS Smoother}
\subsection{Parameter Estimation}
\subsection{Computational Complexity}
\section{Example}
\section{Conclusion}



\appendices
\section{}\label{app:InitialValueProblemProof}
%Appendix one text goes here.
Here we show that the post-synaptic potential $v(\mathbf r, t)$ is a solution to the initial value problem 
\begin{equation}
 dv(\mathbf r, t) + \zeta v(\mathbf r, t)dt = g'(\mathbf r, t) + dW(\mathbf r, t) \quad t\ge0, v(\mathbf r, 0) = v_0
\end{equation}
Consider the function $\kappa(v(\mathbf r, t), t) \in C_{1,2}$ defined as
\begin{equation}
\kappa(v(\mathbf r, t), t) = v(\mathbf r, t)e^{\zeta t}
\end{equation}
 applying Ito's formula we get
\begin{equation}\label{eq:kappafunction}
 d\kappa = e^{\zeta t}g'(\mathbf r, t)dt + e^{\zeta t}\sigma dW(\mathbf r, t)
\end{equation}
Integrating \eqref{eq:kappafunction} over $[0, t]$ gives 
\begin{equation}\label{eq:kappafunctionInteration}
 v(\mathbf r, t)e^{\zeta t}-v_0 = \int_0^te^{\zeta t'}g'(\mathbf r, t')dt' +\int_0^te^{\zeta t'}\sigma dW(\mathbf r, t')
\end{equation}
and multiplying throughout by $e^{-\zeta t}$ gives 
\begin{eqnarray}
\begin{split}
 v(\mathbf r, t) &=&v_0e^{-\zeta t}+ \int_0^te^{-\zeta(t-t')}g'(\mathbf r, t')dt' +\int_0^te^{-\zeta (t-t')}\sigma dW(\mathbf r, t')\\
&=&v_0e^{-\zeta t}+ \int_0^t h(t-t')g'(\mathbf r, t')dt' +\int_0^th(t-t')\sigma dW(\mathbf r, t')
\end{split}
\end{eqnarray}
% Now attach the bibliography
\bibliographystyle{plainnat}
\bibliography{STIDE}
\end{document}
