%
%  untitled
%
%  Created by Dean Freestone on 2010-06-28.
%  Copyright (c) 2010 . All rights reserved.
%
\documentclass[a4paper,10pt]{letter}
\usepackage{anysize} 									%<-use this for checking equation length

\marginsize {1.7cm}{1.5cm}{2cm}{2cm} 				%<-use this for checking equation length

\signature{Dean R. Freestone}
\address{Department of Electrical \\
and Electronic Engineering \\ The University of Melbourne \\ Parkville \\ Victoria, 3010 \\ Australia}

\begin{document}

\begin{letter}{Editor/Reviewer \\ PloS Computational Biology}

\opening{Dear Sir or Madam,}


% \emph{Please explain why this manuscript is suitable for publication in PLoS Computational Biology; why will your paper inspire the other members of your field, and how will it drive research forward?}


I am writing to seek publication of our research article titled `A Data-Driven Framework for Neural Field Modelling' in PLoS Computational Biology.

This paper brings several new techniques to Computational Biology, where as the title suggests, we have developed a data-driven framework for generating neural field models. This paper is timely as it draws together several areas of cutting-edge technologies including:
\begin{itemize}
	\item increases in bandwidths in data acquisition systems, higher resolution recording technology and data storage capabilities that have led to significant advances in neurophysiological recording techniques, 
	\item developments in neural field theory that have led to valuable insights into the mass action of cortical systems,
	\item recent advances in computing power and nonlinear systems theory that have made possible state estimation from nonlinear spatio-temporal systems.
\end{itemize}

This work provides an essential link, bridging the data-driven and theoretically-driven areas of research. We hope that this work will help bring together these previously separated experimental and theoretical communities by providing a framework where both new and existing hypotheses may be tested.

The novel aspects of this paper include:
\begin{itemize}
	\item the representation of a continuum neural field model as reduced finite-dimensional state-space model,
	\item derivations for the conditions for placement and geometry of sensors/electrodes to sufficiently sample a neural field to capture significant cortical dynamics,
	\item conditions for the model reduction to represent the neural field as a finite-dimensional state-space model,
	\item an implementation of the unscented Rauch-Tung-Striebel smoother for state estimation,
	\item a derivation for an algorithm for estimating intracortical connectivity and synaptic dynamics using the neural field equations.
\end{itemize}

Due to the novel aspects of the model reduction and estimation procedure, we have taken the option of preceding the results by the methods section.

Thank you for your time and consideration in reviewing this article. Please contact us if you have any questions regarding any aspects of this paper. I look forward to hearing the outcome of the review soon.

\closing{Best Regards,}

% The authors.
% 
% (Dean Freestone, Parham Aram, Mike Dewar, Ken Scerri, David Grayden and Visakan Kadirkamanathan.)
\end{letter}
\end{document}
