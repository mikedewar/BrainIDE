% Template for PLoS
% Version 1.0 January 2009
%
% To compile to pdf, run:
% latex plos.template
% bibtex plos.template
% latex plos.template
% latex plos.template
% dvipdf plos.template

\documentclass[10pt]{article}

% amsmath package, useful for mathematical formulas
\usepackage{amsmath}
% amssymb package, useful for mathematical symbols
\usepackage{amssymb}

% graphicx package, useful for including eps and pdf graphics
% include graphics with the command \includegraphics
\usepackage{graphicx}

% cite package, to clean up citations in the main text. Do not remove.
\usepackage{cite}

\usepackage{color} 

% Use doublespacing - comment out for single spacing
%\usepackage{setspace} 
%\doublespacing


% Text layout
\topmargin 0.0cm
\oddsidemargin 0.5cm
\evensidemargin 0.5cm
\textwidth 16cm 
\textheight 21cm

% Bold the 'Figure #' in the caption and separate it with a period
% Captions will be left justified
\usepackage[labelfont=bf,labelsep=period,justification=raggedright]{caption}

% Use the PLoS provided bibtex style
\bibliographystyle{plos2009}

% Remove brackets from numbering in List of References
\makeatletter
\renewcommand{\@biblabel}[1]{\quad#1.}
\makeatother


% Leave date blank
\date{}

\pagestyle{myheadings}
%% ** EDIT HERE **


%% ** EDIT HERE **
%% PLEASE INCLUDE ALL MACROS BELOW

%% END MACROS SECTION

\begin{document}

% Title must be 150 characters or less
\begin{flushleft}
{\Large
\textbf{A Data Driven Framework for Patient-Specific Neural Field Modelling}
}
% Insert Author names, affiliations and corresponding author email.
\\
Dean R. Freestone$^{1,2,3}$, 
Parham Aram$^{4}$, 
Michael Dewar$^{5,\ast}$,
Kenneth Scerri$^{6,\ast}$,
David B. Grayden$^{1,\ast}$,
Visakan Kadirkamanathan$^{4}$
\\
\bf{1} Department of Electrical and Electronic Engineering, University of Melbourne, Melbourne, VIC, Australia
\\
\bf{2} The Bionic Ear Institute, East Melbourne, VIC, Australia
\\
\bf{3} Institute for Adaptive and Neural Computation, University of Edinburgh, Edinburgh, UK
\\
\bf{4} Department of Automatic Control and Systems Engineering, University of Sheffield, Sheffield, UK
\\
\bf{5} Department of Applied Physics and Applied Mathematics, Columbia University, New York, NY, USA
\\
\bf{6} Department of Systems and Control Engineering, University of Malta, Msida, MSD, Malta
\\
$\ast$ E-mail: dfreestone@bionicear.org
\end{flushleft}

% Please keep the abstract between 250 and 300 words
\section*{Abstract}
This paper presents a framework for creating patient-specific neural field models from electrophysiological data. The Wilson and Cowen or Amari style neural field equations are used to form a parametric model, where the parameters are estimated from data. To illustrate the estimation framework, data is generated using the neural field equations incorporating modelled sensors, so a comparison can be made between estimated and true parameters. To facilitate state and parameter estimation, we introduce a method to reduce the continuum neural field model, using a basis function decomposition, to form a finite-dimensional state-space model. Spatial frequency analysis methods are introduced for model selection by systematically specifying the basis function configuration required to capture the dominant characteristics of the neural field. The estimation procedure consists of a two-stage iterative algorithm incorporating the unscented Rauch-Tung-Striebel smoother for state estimation and a least squares algorithm for parameter estimation. The results show that it is possible to reconstruct the neural field and estimate intracortical connectivity and synaptic dynamics with the proposed framework. The results also illustrate the loss of high spatial frequency information as the cost incurred by the model reduction procedure. This framework provides a link between patient-specific neurophysiological data and theoretical neural fields models. This link may lead to greater understanding of cortical dynamics at the meso/macroscopic scale where diseases such as epilepsy are manifested.

% Please keep the Author Summary between 150 and 200 words
% Use first person. PLoS ONE authors please skip this step. 
% Author Summary not valid for PLoS ONE submissions.   
\section*{Author Summary}

\section*{Introduction}
Generating physiologically plausible neural field models are of great importance for studying brain dynamics at the meso/macroscopic scale. While our understanding of the function of neurons is well developed, the overall behaviour of the brain's meso and macro-scale dynamics remains largely theoretical. Understanding the brain at this level is extremely important since it is at this scale that pathologies such as epilepsy, Parkinson's disease and schizophrenia are manifested. 

Mathematical neural field models provide insights into the underlying physics and dynamics of electroencephalography (EEG) and magnetoencephalography (MEG) (see \cite{Deco2008,David2003} for recent reviews). These models have demonstrated possible mechanisms for the genesis of neural rhythms (such as the alpha and gamma rhythms) \cite{Liley1999,RENNIE2000}, epileptic seizure generation \cite{DaSilva2003,Suffczynski2004,Wendling2005} and insights into other pathologies \cite{Moran2008,Schiff2009} that would be difficult to gain from experimental data alone. 

Unfortunately, the use of these models in the clinic has been limited, since they are constructed for ``general'' brain dynamics whereas pathologies almost always have unique underlying patient-specific causes. Patient-specific data from electrophysiological recordings is readily available in the clinical setting, particularly from epilepsy surgery patients, suggesting an opportunity to make the patient-specific link to models of cortical dynamics. Furthermore, recent technological advances have driven an increased level of sophistication in recording techniques, with dramatic increases in spatial and temporal sampling~\cite{Brinkmann2009}. However, the meso/macroscopic neural dynamic state is not directly observable in EEG data, making predictions of the underlying physiology inherently difficult.

For models to be clinically viable, they must be patient-specific. A possible approach to achieve this would be to fit a general continuum neural field model, like the Wilson and Cohen (WC)~\cite{Wilson1973} or Amari~\cite{Amari1977} models or a neural mass model like the Jansen and Ritt model~\cite{Jansen1995}, to patient-specific EEG data. Fitting the neural models to individuals is a highly non-trivial task and, until very recently, has not been reported in the literature. 

An estimation framework for neural field models known as dynamical causal modelling (DCM) \cite{David2003,David2006} has recently been proposed for studying evoked potential dynamics. Via a Bayesian inference scheme, DCM estimates the long range (cortico-cortical) connectivity structure between the specific isolated brain regions that best explains a given data set using the Jansen and Ritt equations. Another recent publication describing a parameter estimation method with a neural field model used an unscented Kalman filter with the WC neural field equations~\cite{schiff2008kalman}. This work takes a system theoretic approach to the neural estimation problem, successfully demonstrating that it is possible to perform state estimation of modified WC equations. This marks the first step in what has the potential to revolutionise the treatment of many neurological diseases where therapeutic electrical stimulation is viable.

We present an extension to the work of Schiff and Sauer~\cite{schiff2008kalman} by establishing a framework for estimating the state of the WC equations for larger scale (more space) systems via a systematic model reduction procedure. In addition, a method is presented for estimating the connectivity structure and the synaptic time dynamics. Until now, model-based estimation of local intracortical connectivity has not been reported in the literature (to the best of the authors' knowledge). Our work builds on recent work which shows that it is possible to estimate local coupling of spatio-temporal systems using techniques from control systems theory and machine learning~\cite{Dewar2009}. The key development of this previous work was to represent the spatiotemporal system as a standard state-space model, with the number of states independent of the number of observations (recording electrodes in this case). In addition, the appropriate model selection tools have been developed~\cite{Scerri2009} allowing for the application of the technique to neural fields. 

Modelling the neural dynamics within this framework has a distinct advantage over the more standard multivariate auto-regressive (MVAR) models: the number of parameters to define the spatial connectivity is considerably smaller than the number of AR coefficients typically required to achieve the required model complexity. 

In this paper, we demonstrate for the first time how intracortical connectivity can be inferred from data, based on a variant of the WC neural field model~\cite{Wilson1973}. This work provides a fundamental link between the theoretical advances in neural field modelling and patient-specific data. The paper proceeds by first deriving the continuum neural field equations in Section~\ref{NeuralModelSection}. Then a finite dimensional neural field model is derived. The model is reduced by approximating the neural field using a set of continuous basis functions, weighted by a finite dimensional state vector. Section~\ref{SpectralAnalysisSection} establishes conditions using spatial frequency analysis for both sensor and basis function spacing and width, such that the dominant dynamics of the neural field can be represented by the reduced model. The state and parameter estimation procedure is described in Section~\ref{StateAndParameterEstimationSection}. The results for the spatial frequency analysis and parameter estimation are then presented in Section~\ref{ResultsSection}. The implications and limitations of this framework are discussed in Section~\ref{DiscussionSection} along with planned future developments.

% Results and Discussion can be combined.
\section*{Results}

\subsection*{Subsection 1}

\subsection*{Subsection 2}

\section*{Discussion}

% You may title this section "Methods" or "Models". 
% "Models" is not a valid title for PLoS ONE authors. However, PLoS ONE
% authors may use "Analysis" 
\section*{Materials and Methods}

% Do NOT remove this, even if you are not including acknowledgments
\section*{Acknowledgments}


%\section*{References}
% The bibtex filename
\bibliography{template}

\section*{Figure Legends}
%\begin{figure}[!ht]
%\begin{center}
%%\includegraphics[width=4in]{figure_name.2.eps}
%\end{center}
%\caption{
%{\bf Bold the first sentence.}  Rest of figure 2  caption.  Caption 
%should be left justified, as specified by the options to the caption 
%package.
%}
%\label{Figure_label}
%\end{figure}


\section*{Tables}
%\begin{table}[!ht]
%\caption{
%\bf{Table title}}
%\begin{tabular}{|c|c|c|}
%table information
%\end{tabular}
%\begin{flushleft}Table caption
%\end{flushleft}
%\label{tab:label}
% \end{table}

\end{document}

