%
%  untitled
%
%  Created by Dean Freestone on 2010-06-28.
%  Copyright (c) 2010 . All rights reserved.
%
\documentclass[a4paper,10pt]{letter}
\usepackage{anysize} 									%<-use this for checking equation length

\marginsize {1.7cm}{1.5cm}{1cm}{1cm} 				%<-use this for checking equation length

\signature{Dean R. Freestone}
\address{Department of Electrical \\
and Electronic Engineering \\ The University of Melbourne \\ Victoria, 3010 \\ Australia}

\begin{document}
\vspace{-1cm}
\begin{letter}{Editor/Reviewer \\ PLoS Computational Biology}

\opening{Dear Sir or Madam,}


% \emph{Please explain why this manuscript is suitable for publication in PLoS Computational Biology; why will your paper inspire the other members of your field, and how will it drive research forward?}


I am writing to seek publication of our research article titled `A Data-Driven Framework for Neural Field Modelling' in PLoS Computational Biology.

This paper brings several new techniques to the field of computational biology, where as the title suggests, we have developed a data-driven framework for estimating the parameters of a neural field model from electrophysiological data. Our aim in this work is to model the intracortical connectivity and synaptic dynamics of an individual patient or experimental subject. These subject-specific neural field models will provide new avenues for the application of engineering techniques to neural systems, particularly to epilepsy and other neurological disorders, and to brain-computer interfaces.

This paper is timely as it draws together and is built upon several cutting-edge technologies including:
\begin{itemize}
	\item increases in bandwidths in data acquisition systems, higher electrode technology and data storage capabilities that have led to significant advances in neurophysiological recording techniques, 
	\item developments in neural field theory that have led to valuable insights into the mass action of cortical systems,
	\item recent advances in computing power and nonlinear systems theory that have made possible state estimation from large-scale nonlinear spatiotemporal systems.
\end{itemize}

This work provides an essential link between data-driven and theoretically-driven areas of research. We hope that this work will help to bring together the experimental and theoretical communities by providing a framework upon which both new and existing hypotheses in computational neuroscience may be tested.

The novel aspects of this paper include:
\begin{itemize}
	\item the representation of a continuum neural field model as a reduced finite-dimensional state-space model and the necessary conditions for representing the neural field by a basis function decomposition;
	\item derivations of the necessary conditions for placement and geometry of sensors/electrodes to sufficiently sample a neural field in order to capture the significant cortical dynamics;
	\item an implementation of the unscented Rauch-Tung-Striebel smoother for state estimation of a spatiotemporal neural system;
	\item a derivation of an algorithm for estimating intracortical connectivity and synaptic dynamics using the neural field equations.
\end{itemize}

Due to the novel aspects of the model reduction and estimation procedure, we have chosen to precede the Results section with the Methods section.

Thank you for your time and consideration in reviewing this article. We look forward to hearing the outcome of the review.

\closing{Best Regards,}

% The authors.
% 
% (Dean Freestone, Parham Aram, Mike Dewar, Ken Scerri, David Grayden and Visakan Kadirkamanathan.)
\end{letter}
\end{document}
