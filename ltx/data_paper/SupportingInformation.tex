%
%  untitled
%
%  Created by Parham Aram on 2010-07-14.
%  Copyright (c) 2010 . All rights reserved.
%
\documentclass[]{article}

% Use utf-8 encoding for foreign characters
\usepackage[utf8]{inputenc}

% Setup for fullpage use
\usepackage{fullpage}

% Uncomment some of the following if you use the features
%
% Running Headers and footers
%\usepackage{fancyhdr}

% Multipart figures
%\usepackage{subfigure}

% More symbols
\usepackage{amsmath}
\usepackage{amssymb}
%\usepackage{latexsym}

% Surround parts of graphics with box
\usepackage{boxedminipage}

% Package for including code in the document
\usepackage{listings}

% If you want to generate a toc for each chapter (use with book)
\usepackage{minitoc}

% This is now the recommended way for checking for PDFLaTeX:
\usepackage{ifpdf}

%\newif\ifpdf
%\ifx\pdfoutput\undefined
%\pdffalse % we are not running PDFLaTeX
%\else
%\pdfoutput=1 % we are running PDFLaTeX
%\pdftrue
%\fi

\ifpdf
\usepackage[pdftex]{graphicx}
\else
\usepackage{graphicx}
\fi
% \title{S1: Extended Derivations for `A Data Driven Framework for Patient-Specific Neural Field Modelling'}
% \author{Dean R. Freestone$^{1,2,3,\ast}$, 
% Parham Aram$^{4}$, 
% Michael Dewar$^{5}$,\\
% Kenneth Scerri$^{6}$,
% David B. Grayden$^{1}$,
% Visakan Kadirkamanathan$^{4}$}


\begin{document}

\ifpdf
\DeclareGraphicsExtensions{.pdf, .jpg, .tif}
\else
\DeclareGraphicsExtensions{.eps, .jpg}
\fi

\renewcommand{\theequation}{S1.\arabic{equation}}
\section*{Product of two $n$-dimensional Gaussian functions}\label{sec:GaussianProduct} 
In this section, we provide a derivation for the product of two n-dimensional Gaussian basis functions.
This derivation is used in the  calculation of $\nabla 	q$. Consider two
Gaussian basis functions
\begin{equation}\label{eq:n_dimensional_Gaussian1}
 \varphi_i(\mathbf r)=\mathrm{exp}\left({-\frac{1}{\sigma_i^2} (\mathbf r-\boldsymbol \mu_i)^\top(\mathbf r-\boldsymbol \mu_i})\right)
\end{equation}
and 
\begin{equation}\label{eq:n_dimensional_Gaussian2}
\varphi_j(\mathbf r)=\mathrm{exp}\left({-\frac{1}{\sigma_j^2} (\mathbf r-\boldsymbol \mu_j)^\top(\mathbf r-\boldsymbol \mu_j})\right).
\end{equation}
the product of two Gauusian basis functions is given by
\begin{equation}
 \varphi_i(\mathbf r)\varphi_j(\mathbf r)=\mathrm{exp}-\left({\frac{1}{\sigma_i^2} (\mathbf r-\boldsymbol \mu_i)^\top(\mathbf r-\boldsymbol\mu_i)+{\frac{1}{\sigma_j^2} (\mathbf r-\boldsymbol \mu_j)^\top(\mathbf r-\boldsymbol\mu_j)}}\right)
\end{equation}
the product is expanded to give
\begin{equation}
\begin{array}{ccc}
 
 \varphi_i(\mathbf r)\varphi_j(\mathbf r)&=&\mathrm{exp}-\left(\frac{(\sigma_i^2+\sigma_j^2)\left[\mathbf r^\top\mathbf r-2\mathbf r^\top \frac{\sigma_j^2\boldsymbol\mu_i+\sigma_i^2\boldsymbol\mu_j}{\sigma_i^2+\sigma_j^2}+\frac{(\sigma_j^2\boldsymbol\mu_i+\sigma_i^2\boldsymbol\mu_j)^\top(\sigma_j^2\boldsymbol\mu_i+\sigma_i^2\boldsymbol\mu_j)}{(\sigma_i^2+\sigma_j^2)^2}   \right]  +\frac{\sigma_i^2\sigma_j^2}{\sigma_i^2+\sigma_j^2}(\boldsymbol \mu_i-\boldsymbol\mu_j)^\top(\boldsymbol \mu_i-\boldsymbol\mu_j) }{\sigma_i^2\sigma_j^2}\right)\\
&=&\mathrm{exp}\left(-\frac{(\sigma_i^2+\sigma_j^2)\left[(\mathbf r-\frac{\sigma_j^2\boldsymbol\mu_i+\sigma_i^2\boldsymbol\mu_j}{\sigma_i^2+\sigma_j^2})^\top(\mathbf r-\frac{\sigma_j^2\boldsymbol\mu_i+\sigma_i^2\boldsymbol\mu_j}{\sigma_i^2+\sigma_j^2})\right]  }{\sigma_i^2\sigma_j^2}\right)
\times\mathrm{exp}\left(-\frac{(\boldsymbol \mu_i-\boldsymbol\mu_j)^\top(\boldsymbol \mu_i-\boldsymbol\mu_j)}{\sigma_i^2+\sigma_j^2}\right)
\end{array}
\end{equation}
therefore we have
\begin{equation}
 \varphi_i(\mathbf r)\varphi_j(\mathbf r)=c_{i,j}\times\mathrm{exp}\left({-\frac{1}{\sigma^2} (\mathbf r-\boldsymbol \mu)^\top(\mathbf r-\boldsymbol\mu)}\right)
\end{equation}
where
\begin{equation}
 c_{i,j}=\mathrm{exp}\left(-\frac{(\boldsymbol \mu_i-\boldsymbol\mu_j)^\top(\boldsymbol \mu_i-\boldsymbol\mu_j)}{\sigma_i^2+\sigma_j^2}\right) \quad \sigma^2=\frac{\sigma_i^2\sigma_j^2}{\sigma_i^2+\sigma_j^2}
\end{equation}
and
\begin{equation}
 \boldsymbol\mu=\frac{\sigma_j^2\boldsymbol\mu_i+\sigma_i^2\boldsymbol\mu_j}{\sigma_i^2+\sigma_j^2}
\end{equation}
\section*{Derivative of the firing rate}\label{sec:FiringrateDerivative} 
Here we find the derivative of the activation function which is used to compute  $\nabla q$. The sigmoidal activation function which relates firing rate of the presynaptic neurons to the post-synaptic membrane potential is given by
\begin{equation}
	\label{ActivationFunction} f\left( v\left( \mathbf{r}', t \right) \right) = \frac{f_{max}}{1 + \exp \left( \varsigma \left( v_0 - v\left(\mathbf{r}',t\right) \right) \right)}. 
\end{equation}
Derivative of the sigmoidal activation function can be written in a simple form as
\begin{eqnarray}
 \frac{\sigma f}{\sigma v}&=& \frac{\varsigma f_{max}}{\left(1 + \exp \left( \varsigma \left( v_0 - v\left(\mathbf{r}',t\right) \right) \right)\right)^2} \times \exp \left( \varsigma \left( v_0 - v\left(\mathbf{r}',t\right) \right) \right) \nonumber \\
&=&\frac{\varsigma f_{max}}{1 + \exp \left( \varsigma \left( v_0 - v\left(\mathbf{r}',t\right) \right) \right)} \times \left(1-\frac{1}{1 + \exp \left( \varsigma \left( v_0 - v\left(\mathbf{r}',t\right) \right) \right)}\right) \nonumber \\
&=& \varsigma f\left( v\left( \mathbf{r}', t \right) \right)\left( 1-\frac{f\left( v\left( \mathbf{r}', t \right) \right)}{f_{max}}\right)
\end{eqnarray}



\bibliographystyle{plain}
\bibliography{}
\end{document}
