%
%  untitled
%
%  Created by Parham Aram on 2010-07-14.
%  Copyright (c) 2010 . All rights reserved.
%
\documentclass[]{article}

% Use utf-8 encoding for foreign characters
\usepackage[utf8]{inputenc}

% Setup for fullpage use
\usepackage{fullpage}

% Uncomment some of the following if you use the features
%
% Running Headers and footers
%\usepackage{fancyhdr}

% Multipart figures
%\usepackage{subfigure}

% More symbols
\usepackage{amsmath}
\usepackage{amssymb}
%\usepackage{latexsym}

% Surround parts of graphics with box
\usepackage{boxedminipage}

% Package for including code in the document
\usepackage{listings}

% If you want to generate a toc for each chapter (use with book)
\usepackage{minitoc}

% This is now the recommended way for checking for PDFLaTeX:
\usepackage{ifpdf}

%\newif\ifpdf
%\ifx\pdfoutput\undefined
%\pdffalse % we are not running PDFLaTeX
%\else
%\pdfoutput=1 % we are running PDFLaTeX
%\pdftrue
%\fi

\ifpdf
\usepackage[pdftex]{graphicx}
\else
\usepackage{graphicx}
\fi
% \title{S1: Extended Derivations for `A Data Driven Framework for Patient-Specific Neural Field Modelling'}
% \author{Dean R. Freestone$^{1,2,3,\ast}$, 
% Parham Aram$^{4}$, 
% Michael Dewar$^{5}$,\\
% Kenneth Scerri$^{6}$,
% David B. Grayden$^{1}$,
% Visakan Kadirkamanathan$^{4}$}


\begin{document}

\ifpdf
\DeclareGraphicsExtensions{.pdf, .jpg, .tif}
\else
\DeclareGraphicsExtensions{.eps, .jpg}
\fi

\renewcommand{\theequation}{S1.\arabic{equation}}
\section*{Product of two $n$-dimensional Gaussian functions}\label{sec:GaussianProduct} 
In this section, we provide a derivation for the product of two n-dimensional Gaussian basis functions.
This derivation is used in the  calculation of $\nabla 	q$. Consider two
Gaussian basis functions
\begin{equation}\label{eq:n_dimensional_Gaussian1}
 \varphi_i(\mathbf r)=\mathrm{exp}\left({-\frac{1}{\sigma_i^2} (\mathbf r-\boldsymbol \mu_i)^\top(\mathbf r-\boldsymbol \mu_i})\right)
\end{equation}
and 
\begin{equation}\label{eq:n_dimensional_Gaussian2}
\varphi_j(\mathbf r)=\mathrm{exp}\left({-\frac{1}{\sigma_j^2} (\mathbf r-\boldsymbol \mu_j)^\top(\mathbf r-\boldsymbol \mu_j})\right).
\end{equation}
the product of two Gauusian basis functions is given by
\begin{equation}
 \varphi_i(\mathbf r)\varphi_j(\mathbf r)=\mathrm{exp}-\left({\frac{1}{\sigma_i^2} (\mathbf r-\boldsymbol \mu_i)^\top(\mathbf r-\boldsymbol\mu_i)+{\frac{1}{\sigma_j^2} (\mathbf r-\boldsymbol \mu_j)^\top(\mathbf r-\boldsymbol\mu_j)}}\right)
\end{equation}
the product is expanded to give
\begin{equation}
\begin{array}{ccc}
 
 \varphi_i(\mathbf r)\varphi_j(\mathbf r)&=&\mathrm{exp}-\left(\frac{(\sigma_i^2+\sigma_j^2)\left[\mathbf r^\top\mathbf r-2\mathbf r^\top \frac{\sigma_j^2\boldsymbol\mu_i+\sigma_i^2\boldsymbol\mu_j}{\sigma_i^2+\sigma_j^2}+\frac{(\sigma_j^2\boldsymbol\mu_i+\sigma_i^2\boldsymbol\mu_j)^\top(\sigma_j^2\boldsymbol\mu_i+\sigma_i^2\boldsymbol\mu_j)}{(\sigma_i^2+\sigma_j^2)^2}   \right]  +\frac{\sigma_i^2\sigma_j^2}{\sigma_i^2+\sigma_j^2}(\boldsymbol \mu_i-\boldsymbol\mu_j)^\top(\boldsymbol \mu_i-\boldsymbol\mu_j) }{\sigma_i^2\sigma_j^2}\right)\\
&=&\mathrm{exp}\left(-\frac{(\sigma_i^2+\sigma_j^2)\left[(\mathbf r-\frac{\sigma_j^2\boldsymbol\mu_i+\sigma_i^2\boldsymbol\mu_j}{\sigma_i^2+\sigma_j^2})^\top(\mathbf r-\frac{\sigma_j^2\boldsymbol\mu_i+\sigma_i^2\boldsymbol\mu_j}{\sigma_i^2+\sigma_j^2})\right]  }{\sigma_i^2\sigma_j^2}\right)
\times\mathrm{exp}\left(-\frac{(\boldsymbol \mu_i-\boldsymbol\mu_j)^\top(\boldsymbol \mu_i-\boldsymbol\mu_j)}{\sigma_i^2+\sigma_j^2}\right)
\end{array}
\end{equation}
therefore we have
\begin{equation}
 \varphi_i(\mathbf r)\varphi_j(\mathbf r)=c_{i,j}\times\mathrm{exp}\left({-\frac{1}{\sigma^2} (\mathbf r-\boldsymbol \mu)^\top(\mathbf r-\boldsymbol\mu)}\right)
\end{equation}
where
\begin{equation}
 c_{i,j}=\mathrm{exp}\left(-\frac{(\boldsymbol \mu_i-\boldsymbol\mu_j)^\top(\boldsymbol \mu_i-\boldsymbol\mu_j)}{\sigma_i^2+\sigma_j^2}\right) \quad \sigma^2=\frac{\sigma_i^2\sigma_j^2}{\sigma_i^2+\sigma_j^2}
\end{equation}
and
\begin{equation}
 \boldsymbol\mu=\frac{\sigma_j^2\boldsymbol\mu_i+\sigma_i^2\boldsymbol\mu_j}{\sigma_i^2+\sigma_j^2}
\end{equation}
\section*{Derivative of the firing rate}\label{sec:FiringrateDerivative} 
Here we find the derivative of the activation function which is used to compute  $\nabla q$. The sigmoidal activation function which relates firing rate of the presynaptic neurons to the post-synaptic membrane potential is given by
\begin{equation}
	\label{ActivationFunction} f\left( v\left( \mathbf{r}', t \right) \right) = \frac{f_{max}}{1 + \exp \left( \varsigma \left( v_0 - v\left(\mathbf{r}',t\right) \right) \right)}. 
\end{equation}
Derivative of the sigmoidal activation function can be written in a simple form as
\begin{eqnarray}
 \frac{\sigma f}{\sigma v}&=& \frac{\varsigma f_{max}}{\left(1 + \exp \left( \varsigma \left( v_0 - v\left(\mathbf{r}',t\right) \right) \right)\right)^2} \times \exp \left( \varsigma \left( v_0 - v\left(\mathbf{r}',t\right) \right) \right) \nonumber \\
&=&\frac{\varsigma f_{max}}{1 + \exp \left( \varsigma \left( v_0 - v\left(\mathbf{r}',t\right) \right) \right)} \times \left(1-\frac{1}{1 + \exp \left( \varsigma \left( v_0 - v\left(\mathbf{r}',t\right) \right) \right)}\right) \nonumber \\
&=& \varsigma f\left( v\left( \mathbf{r}', t \right) \right)\left( 1-\frac{f\left( v\left( \mathbf{r}', t \right) \right)}{f_{max}}\right)
\end{eqnarray}
\section*{$\Xi$-variables}\label{sec:Xivariables} 
In this section, we provide derivation for $\Xi$-variables, which we use in the M-step. From equation 47 of the main text we have
\begin{eqnarray}\label{eq:Qfunction}
  \ln p(\mathbf x_{t+1} | \mathbf x_t;\boldsymbol\theta,\xi)&=&\ln \alpha-\mathbf x_{t+1}^\top\boldsymbol\Sigma_e^{-1}\mathbf x_{t+1}+2\mathbf x_{t+1}^\top\boldsymbol\Sigma_e^{-1}q( \mathbf x_t)\boldsymbol\theta+2\xi \mathrm{tr} \left\lbrace \mathbf x_t\mathbf x_{t+1}^\top\boldsymbol\Sigma_e^{-1}\right\rbrace \nonumber \\
&&-\boldsymbol\theta^\top q^\top(\mathbf x_t)\boldsymbol\Sigma_e^{-1}q(\mathbf x_t)\boldsymbol\theta-2\xi \mathbf x_t^\top\boldsymbol\Sigma_e^{-1}q(\mathbf x_t)\boldsymbol\theta-\xi^2\mathrm{tr}\left\lbrace \mathbf x_t \mathbf x_t^\top\boldsymbol\Sigma_e^{-1}\right\rbrace. 
\end{eqnarray}
substituting for $q(\mathbf x_t)$ using the approximation given in equation 40 of the main text we calculate each $\Xi$-variable as following
\subsection*{Calculating $\Xi_0$}
\begin{eqnarray}\label{eq:Xi0Derivation1}
 \mathbf x_{t+1}^\top\boldsymbol\Sigma_e^{-1}q( \mathbf x_t) &\approx&  \mathbf x_{t+1}^\top\boldsymbol\Sigma_e^{-1}\left[q(\mathbf {\hat x}_t)+\int_\Omega \boldsymbol{\Psi}(\mathbf{r}')\boldsymbol \phi^\top(\mathbf r') (\mathbf x_t - \mathbf  {\hat x}_t)f'(\boldsymbol \phi^\top(\mathbf r')\mathbf {\hat x}_t) d\mathbf{r}'\right] \nonumber \\
&=&\mathbf x_{t+1}^\top\boldsymbol\Sigma_e^{-1}q(\mathbf {\hat x}_t)+\mathbf x_{t+1}^\top\boldsymbol\Sigma_e^{-1}\int_\Omega \boldsymbol{\Psi}(\mathbf{r}')\boldsymbol \phi^\top(\mathbf r') (\mathbf x_t - \mathbf  {\hat x}_t)f'(\boldsymbol \phi^\top(\mathbf r')\mathbf {\hat x}_t) d\mathbf{r}'
\end{eqnarray}
noting that $\phi^\top(\mathbf r') (\mathbf x_t - \mathbf  {\hat x}_t)$ is a scalar we can write \ref{eq:Xi0Derivation1} as 
\begin{eqnarray}\label{eq:Xi0Derivation2}
 \mathbf x_{t+1}^\top\boldsymbol\Sigma_e^{-1}q( \mathbf x_t) &\approx& \mathbf x_{t+1}^\top\boldsymbol\Sigma_e^{-1}q(\mathbf {\hat x}_t)+\int_\Omega \boldsymbol \phi^\top(\mathbf r') (\mathbf x_t - \mathbf  {\hat x}_t)\mathbf x_{t+1}^\top\boldsymbol\Sigma_e^{-1}    \boldsymbol{\Psi}(\mathbf{r}')f'(\boldsymbol \phi^\top(\mathbf r')\mathbf {\hat x}_t) d\mathbf{r}'
\end{eqnarray}
taking expected value of \ref{eq:Xi0Derivation2} we get 
\begin{eqnarray}\label{eq:Xi0Derivation3}
 \mathbf E_{\boldsymbol (\theta',\xi')}\left[\mathbf x_{t+1}^\top\boldsymbol\Sigma_e^{-1}q( \mathbf x_t)\right] &\approx& \mathbf E_{\boldsymbol (\theta',\xi')}\left[\mathbf x_{t+1}^\top\right]\boldsymbol\Sigma_e^{-1}q(\mathbf {\hat x}_t)+\int_\Omega \boldsymbol \phi^\top(\mathbf r')  \mathbf E_{\boldsymbol (\theta',\xi')}\left[(\mathbf x_t - \mathbf  {\hat x}_t)\mathbf x_{t+1}^\top\right]\boldsymbol\Sigma_e^{-1}    \boldsymbol{\Psi}(\mathbf{r}')f'(\boldsymbol \phi^\top(\mathbf r')\mathbf {\hat x}_t) d\mathbf{r}' \nonumber \\
&=&\mathbf {\hat x}_{t+1}^\top\boldsymbol\Sigma_e^{-1}q(\mathbf {\hat x}_t)+\int_\Omega \boldsymbol \phi^\top(\mathbf r') \mathbf M_{t+1}\boldsymbol\Sigma_e^{-1}    \boldsymbol{\Psi}(\mathbf{r}')f'(\boldsymbol \phi^\top(\mathbf r')\mathbf {\hat x}_t) d\mathbf{r}'
\end{eqnarray}
summing over $t \in \left\lbrace 0, \dots, T-1\right\rbrace $ we get
\begin{equation}
\Xi_0=\sum_{t=0}^{T-1}\left[ \mathbf{\hat x}_{t+1}^\top\boldsymbol\Sigma_e^{-1}q(\mathbf{\hat x}_t)+\int_\Omega\boldsymbol \phi^\top(\mathbf r') \mathbf M_{t+1} \boldsymbol\Sigma_e^{-1}  \boldsymbol{\Psi}(\mathbf{r}') f'(\boldsymbol \phi^\top(\mathbf r')\mathbf {\hat x}_t) d\mathbf{r}'\right] \\	
\end{equation}
\subsection*{Calculating $\Xi_1$}
We have
\begin{equation}\label{eq:Xi1}
 \Xi_1=\mathbf E_{(\theta',\xi')}\left[\sum_{t=0}^{T-1}\mathbf x_t\mathbf x_{t+1}^\top\right]
\end{equation}
To find an expression for $\Xi_1$ in terms of smoother outputs we start by expanding the cross-covariance matrix as following
\begin{eqnarray}
 \mathbf M_{t+1}&=&\mathbf E_{(\theta',\xi')}\left[(\mathbf x_t-\mathbf{\hat x}_t)(\mathbf x_{t+1}-\mathbf{\hat x}_{t+1})^\top\right] \nonumber \\
&=&\mathbf E_{(\theta',\xi')}\left[\mathbf x_t\mathbf x_{t+1}^\top\right]-\mathbf E_{(\theta',\xi')}\left[\mathbf x_t\mathbf{\hat x}_{t+1}^\top\right]-\mathbf E_{(\theta',\xi')}\left[\mathbf{\hat x}_t\mathbf x_{t+1}^\top\right]+\mathbf E_{(\theta',\xi')}\left[\mathbf{\hat x}_t\mathbf{\hat x}_{t+1}^\top\right] \nonumber \\
&=&\mathbf E_{(\theta',\xi')}\left[\mathbf x_t\mathbf x_{t+1}^\top\right]-\mathbf {\hat x}_t\mathbf{\hat x}_{t+1}^\top-\mathbf {\hat x}_t\mathbf{\hat x}_{t+1}^\top+\mathbf {\hat x}_t\mathbf{\hat x}_{t+1}^\top \nonumber \\
&=&\mathbf E_{(\theta',\xi')}\left[\mathbf x_t\mathbf x_{t+1}^\top\right]-\mathbf {\hat x}_t\mathbf{\hat x}_{t+1}^\top
\end{eqnarray}
therefore we can write
\begin{equation}\label{eq:Xi1derivation1}
 \mathbf E_{(\theta',\xi')}\left[\mathbf x_t\mathbf x_{t+1}^\top\right]=\mathbf M_{t+1}+\mathbf {\hat x}_t\mathbf{\hat x}_{t+1}^\top
\end{equation}
substituting \ref{eq:Xi1derivation1} in \ref{eq:Xi1} we get
\begin{equation}
 \Xi_1=\sum_{t=0}^{T-1}\left(\mathbf M_{t+1}+\mathbf{\hat x}_t\mathbf{\hat x}_{t+1}^\top\right)
\end{equation}
\subsection*{Calculating $\Xi_2$}
\begin{eqnarray}\label{eq:Xi2derivation1}
 q^\top(\mathbf x_t)\boldsymbol\Sigma_e^{-1}q(\mathbf x_t)&\approx& \left[ q(\mathbf {\hat x}_t)+\int_\Omega \boldsymbol{\Psi}(\mathbf{r}')\boldsymbol \phi^\top(\mathbf r') (\mathbf x_t - \mathbf  {\hat x}_t)f'(\boldsymbol \phi^\top(\mathbf r')\mathbf {\hat x}_t) d\mathbf{r}'\right]^\top  \times \boldsymbol\Sigma_e^{-1}\nonumber \\
&\times&\left[ q(\mathbf {\hat x}_t)+\int_\Omega \boldsymbol{\Psi}(\mathbf{r}')\boldsymbol \phi^\top(\mathbf r') (\mathbf x_t - \mathbf  {\hat x}_t)f'(\boldsymbol \phi^\top(\mathbf r')\mathbf {\hat x}_t) d\mathbf{r}'\right] \nonumber \\
&=&q^\top(\mathbf {\hat x}_t)\boldsymbol\Sigma_e^{-1}q(\mathbf {\hat x}_t)+q^\top(\mathbf {\hat x}_t)\int_\Omega \boldsymbol{\Psi}(\mathbf{r}')\boldsymbol \phi^\top(\mathbf r') (\mathbf x_t - \mathbf  {\hat x}_t)f'(\boldsymbol \phi^\top(\mathbf r')\mathbf {\hat x}_t) d\mathbf{r}'\nonumber\\
&+&\int_\Omega \boldsymbol{\Psi}^\top(\mathbf{r}')\boldsymbol \phi^\top(\mathbf r') (\mathbf x_t - \mathbf  {\hat x}_t)f'(\boldsymbol \phi^\top(\mathbf r')\mathbf {\hat x}_t) d\mathbf{r}'\boldsymbol\Sigma_e^{-1}q(\mathbf {\hat x}_t) \nonumber \\
&+&\int_\Omega \boldsymbol{\Psi}^\top(\mathbf{r}')\boldsymbol \phi^\top(\mathbf r') (\mathbf x_t - \mathbf  {\hat x}_t)f'(\boldsymbol \phi^\top(\mathbf r')\mathbf {\hat x}_t) d\mathbf{r}' \times \boldsymbol  \Sigma_e^{-1} \times \int_\Omega \boldsymbol{\Psi}(\mathbf{r}')\boldsymbol \phi^\top(\mathbf r') (\mathbf x_t - \mathbf  {\hat x}_t)f'(\boldsymbol \phi^\top(\mathbf r')\mathbf {\hat x}_t) d\mathbf{r}' \nonumber \\
&&
\end{eqnarray}
 The last term on the right hand side of equation \ref{eq:Xi2derivation1} can be rewritten as
\begin{eqnarray}
 \int_\Omega \boldsymbol{\Psi}^\top(\mathbf{r}') \boldsymbol\Sigma_e^{-1}f'(\boldsymbol \phi^\top(\mathbf r')\mathbf {\hat x}_t) \boldsymbol \phi^\top(\mathbf r')(\mathbf x_t - \mathbf  {\hat x}_t)d\mathbf{r}' \times\int_\Omega  (\mathbf x_t - \mathbf  {\hat x}_t)^\top\boldsymbol \phi(\mathbf r') \boldsymbol{\Psi}(\mathbf{r}')f'(\boldsymbol \phi^\top(\mathbf r')\mathbf {\hat x}_t) d\mathbf{r}'&& \nonumber \\
=\begin{bmatrix} \int_{\Omega} \boldsymbol \Psi^\top(\mathbf r')\boldsymbol \Sigma_e^{-1}\phi_1(\mathbf r')f'(\boldsymbol\phi^\top(\mathbf r')\mathbf {\hat x}_t)d\mathbf r' &  \dots & \int_{\Omega} \boldsymbol \Psi^\top(\mathbf r')\boldsymbol \Sigma_e^{-1}\phi_i(\mathbf r')f'(\boldsymbol\phi^\top(\mathbf r')\mathbf {\hat x}_t)d\mathbf r'\end{bmatrix}(\mathbf x_t - \mathbf  {\hat x}_t) && \nonumber \\
\times (\mathbf x_t - \mathbf  {\hat x}_t)^\top \begin{bmatrix}\int_{\Omega} \boldsymbol \Psi(\mathbf r')\phi_1(\mathbf r')f'(\boldsymbol\phi^\top(\mathbf r')\mathbf {\hat x}_t)d\mathbf r' \\  \vdots \\ \int_{\Omega} \boldsymbol \Psi(\mathbf r')\phi_i(\mathbf r')f'(\boldsymbol\phi^\top(\mathbf r')\mathbf {\hat x}_t)d\mathbf r'\end{bmatrix}&& \nonumber \\
\end{eqnarray}
therefore \ref{eq:Xi2derivation1} can be written as
\begin{eqnarray}\label{eq:Xi2derivation2}
  q^\top(\mathbf x_t)\boldsymbol\Sigma_e^{-1}q(\mathbf x_t)&\approx&q^\top(\mathbf {\hat x}_t)\boldsymbol\Sigma_e^{-1}q(\mathbf {\hat x}_t)+\Lambda_t(\mathbf r)(\mathbf x_t - \mathbf  {\hat x}_t) (\mathbf x_t - \mathbf  {\hat x}_t)^\top\tilde{\Lambda}_t(\mathbf r)+\tilde{q}(\mathbf x_t)
\end{eqnarray}
where $\Lambda_t(\mathbf r)$ and $\tilde{\Lambda}_t(\mathbf r)$ are defined in equations (54-57) of the main text, and  
\begin{eqnarray}\label{eq:qtilde}
 \tilde{q}(\mathbf x_t)&=&q^\top(\mathbf {\hat x}_t)\int_\Omega \boldsymbol{\Psi}(\mathbf{r}')\boldsymbol \phi^\top(\mathbf r') (\mathbf x_t - \mathbf  {\hat x}_t)f'(\boldsymbol \phi^\top(\mathbf r')\mathbf {\hat x}_t) d\mathbf{r}'\nonumber\\
&+&\int_\Omega \boldsymbol{\Psi}^\top(\mathbf{r}')\boldsymbol \phi^\top(\mathbf r') (\mathbf x_t - \mathbf  {\hat x}_t)f'(\boldsymbol \phi^\top(\mathbf r')\mathbf {\hat x}_t) d\mathbf{r}'\boldsymbol\Sigma_e^{-1}q(\mathbf {\hat x}_t)
\end{eqnarray}
taking expectation from \ref{eq:Xi2derivation2} we have 
\begin{equation}\label{eq:Xi2derivation3}
\mathbf E_{(\theta',\xi')}\left[ q^\top(\mathbf x_t)\boldsymbol\Sigma_e^{-1}q(\mathbf x_t)\right]\approx q^\top(\mathbf {\hat x}_t)\boldsymbol\Sigma_e^{-1}q(\mathbf {\hat x}_t)  +\Lambda_t(\mathbf r)\mathbf E_{(\theta',\xi')}\left[  (\mathbf x_t - \mathbf  {\hat x}_t) (\mathbf x_t - \mathbf  {\hat x}_t)^\top\right]\tilde{\Lambda}_t(\mathbf r)
\end{equation}


Note that the expectation of $\tilde{q}(\mathbf x_t)$ becomes zero since $\mathbf E_{(\theta',\xi')}\left[(\mathbf x_t - \mathbf  {\hat x}_t)\right]= \mathbf 0$, and hence
\begin{equation}
 \Xi_2= \sum_{t=0}^{T-1}\left[q^\top(\mathbf{\hat x}_t)\boldsymbol\Sigma_e^{-1}q(\mathbf{\hat x}_t)+\Lambda_t(\mathbf r)\mathbf P_t \tilde{\Lambda}_t(\mathbf r)\right]
\end{equation}
\subsection*{Calculating $\Xi_3$}
\begin{eqnarray}\label{eq:Xi3derivation1}
 \mathbf x_t^\top\boldsymbol\Sigma_e^{-1}q(\mathbf x_t)&\approx&\mathbf x_t^\top\boldsymbol\Sigma_e^{-1}\left[q(\mathbf {\hat x}_t)+\int_\Omega \boldsymbol{\Psi}(\mathbf{r}')\boldsymbol \phi^\top(\mathbf r') (\mathbf x_t - \mathbf  {\hat x}_t)f'(\boldsymbol \phi^\top(\mathbf r')\mathbf {\hat x}_t) d\mathbf{r}'\right] \nonumber\\
&=&\mathbf x_t^\top\boldsymbol\Sigma_e^{-1}q(\mathbf {\hat x}_t)+\mathbf x_t^\top\boldsymbol\Sigma_e^{-1}\int_\Omega \boldsymbol{\Psi}(\mathbf{r}')\boldsymbol \phi^\top(\mathbf r') (\mathbf x_t - \mathbf  {\hat x}_t)f'(\boldsymbol \phi^\top(\mathbf r')\mathbf {\hat x}_t) d\mathbf{r}' \nonumber \\
&=&\mathbf x_t^\top\boldsymbol\Sigma_e^{-1}q(\mathbf {\hat x}_t)+\int_\Omega \boldsymbol \phi^\top(\mathbf r') (\mathbf x_t - \mathbf  {\hat x}_t)\mathbf x_t^\top\boldsymbol\Sigma_e^{-1}\boldsymbol{\Psi}(\mathbf{r}')f'(\boldsymbol \phi^\top(\mathbf r')\mathbf {\hat x}_t) d\mathbf{r}'
\end{eqnarray}
taking expected value of \ref{eq:Xi3derivation1} we get
\begin{eqnarray}\label{eq:Xi3derivation2}
\mathbf E_{(\theta',\xi')}\left[\mathbf x_t^\top\boldsymbol\Sigma_e^{-1}q(\mathbf x_t)\right]&\approx&\mathbf E_{(\theta',\xi')}\left[\mathbf x_t^\top\right]\boldsymbol\Sigma_e^{-1}+\int_\Omega \boldsymbol \phi^\top(\mathbf r') \mathbf E_{(\theta',\xi')}\left[(\mathbf x_t - \mathbf  {\hat x}_t)\mathbf x_t^\top\right]\boldsymbol\Sigma_e^{-1}\boldsymbol{\Psi}(\mathbf{r}')f'(\boldsymbol \phi^\top(\mathbf r')\mathbf {\hat x}_t) d\mathbf{r}' \nonumber \\
&=&\mathbf{ \hat x}_t^\top\boldsymbol\Sigma_e^{-1}+\int_\Omega \boldsymbol \phi^\top(\mathbf r') \mathbf P_t\boldsymbol\Sigma_e^{-1}\boldsymbol{\Psi}(\mathbf{r}')f'(\boldsymbol \phi^\top(\mathbf r')\mathbf {\hat x}_t) d\mathbf{r}'
\end{eqnarray}
summing over $t \in \left\lbrace 0, \dots, T-1\right\rbrace $ we get
\begin{equation}
\Xi_3=\sum_{t=0}^{T-1}\left[\mathbf{ \hat x}_t^\top\boldsymbol\Sigma_e^{-1}+\int_\Omega \boldsymbol \phi^\top(\mathbf r') \mathbf P_t\boldsymbol\Sigma_e^{-1}\boldsymbol{\Psi}(\mathbf{r}')f'(\boldsymbol \phi^\top(\mathbf r')\mathbf {\hat x}_t) d\mathbf{r}' \right] \\	
\end{equation}
\subsection*{Calculating $\Xi_4$}
We have
\begin{equation}\label{eq:Xi4}
 \Xi_4=\mathbf E_{(\theta',\xi')}\left[\sum_{t=0}^{T-1}\mathbf x_t\mathbf x_{t}^\top\right]
\end{equation}
To find an expression for $\Xi_4$ in terms of smoother outputs we start by expanding the covariance matrix as following
\begin{eqnarray}
 \mathbf P_{t}&=&\mathbf E_{(\theta',\xi')}\left[(\mathbf x_t-\mathbf{\hat x}_t)(\mathbf x_{t}-\mathbf{\hat x}_{t})^\top\right] \nonumber \\
&=&\mathbf E_{(\theta',\xi')}\left[\mathbf x_t\mathbf x_{t}^\top\right]-\mathbf E_{(\theta',\xi')}\left[\mathbf x_t\mathbf{\hat x}_{t}^\top\right]-\mathbf E_{(\theta',\xi')}\left[\mathbf{\hat x}_t\mathbf x_{t}^\top\right]+\mathbf E_{(\theta',\xi')}\left[\mathbf{\hat x}_t\mathbf{\hat x}_{t}^\top\right] \nonumber \\
&=&\mathbf E_{(\theta',\xi')}\left[\mathbf x_t\mathbf x_{t}^\top\right]-\mathbf {\hat x}_t\mathbf{\hat x}_{t}^\top-\mathbf {\hat x}_t\mathbf{\hat x}_{t}^\top+\mathbf {\hat x}_t\mathbf{\hat x}_{t}^\top \nonumber \\
&=&\mathbf E_{(\theta',\xi')}\left[\mathbf x_t\mathbf x_{t}^\top\right]-\mathbf {\hat x}_t\mathbf{\hat x}_{t}^\top
\end{eqnarray}
therefore we can write
\begin{equation}\label{eq:Xi4derivation1}
 \mathbf E_{(\theta',\xi')}\left[\mathbf x_t\mathbf x_{t}^\top\right]=\mathbf P_{t}+\mathbf {\hat x}_t\mathbf{\hat x}_{t}^\top
\end{equation}
substituting \ref{eq:Xi4derivation1} in \ref{eq:Xi4} we get
\begin{equation}
 \Xi_4=\sum_{t=0}^{T-1}\left(\mathbf P_{t}+\mathbf{\hat x}_t\mathbf{\hat x}_{t}^\top\right)
\end{equation}

\bibliographystyle{plain}
\bibliography{}
\end{document}
